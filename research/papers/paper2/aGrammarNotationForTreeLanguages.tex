\documentclass[journal]{IEEEtran}
\usepackage{blindtext}
\usepackage{graphicx}
\usepackage{listings}
\usepackage[superscript,biblabel]{cite}


\hyphenation{op-tical net-works semi-conduc-tor}

\begin{document}

\title{A Grammar Notation for Tree Languages}

\author{Breck~Yunits% <-this % stops a space
\thanks{Breck Yunits is a researcher at Ohayo Computer (breck@ohayo.computer)}% <-this % stops a space
}

\markboth{October~2017 DRAFT}%
{Shell \MakeLowercase{\textit{et al.}}: Bare Demo of IEEEtran.cls for Journals}

\maketitle


\begin{abstract}
%\boldmath
I introduce the core idea of a new grammar notation for formally describing Tree Languages.
\end{abstract}

\IEEEpeerreviewmaketitle

\section{Introduction}

Creating a great programming language is a multi-step process. One step in that process is to decide on syntax and formally define a language in a grammar notation such as BNF. Unfortunately, like the programming languages they describe, these grammar notations are complex and error-prone.

Below I introduce the core idea of a much simpler grammar notation for defining Tree Languages.

\section{A Grammar Notation for Tree Languages}

A Tree Language Grammar is a \textit{double} consisting of a set of Keyword Definitions and a catchall Keyword.

A Keyword Definition is a \textit{double} consisting of a unique keyword identifier and a Grammar.

Everything is encoded in Tree Notation, hence the grammar notation itself is a Tree Language.

\section{Example}

A Tree Language Grammar file for an imagined Tree Language called Tally, with 2 possible recursive keywords \{+, -\} might look like this:

\begin{lstlisting}
Tally
 catchAll error
 keywords
  error
  expression Tally
   words int+
  + expression
  - expression
\end{lstlisting}

A valid program in the Tally language defined by the file above:

\begin{lstlisting}
+ 4 5
 - 1 1
\end{lstlisting}

\section{Conclusion and Future Work}

The introduction above is minimal but shows the core idea: Tree Languages can be formally defined in a simple grammar notation that itself is a Tree Language.

Ohayo Computer has developed a compiler-compiler for these grammar files. Future publications and/or open source releases will delve into the additional features found in the compiler-compiler and its associated grammar.

\end{document}
